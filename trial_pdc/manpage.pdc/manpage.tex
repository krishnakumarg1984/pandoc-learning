% Options for packages loaded elsewhere
\PassOptionsToPackage{unicode}{hyperref}
\PassOptionsToPackage{hyphens}{url}
%
\documentclass[
  12pt,
  a4paper,
]{article}
\usepackage{lmodern}
\usepackage{setspace}
\usepackage{amssymb,amsmath}
\usepackage{ifxetex,ifluatex}
\ifnum 0\ifxetex 1\fi\ifluatex 1\fi=0 % if pdftex
  \usepackage[T1]{fontenc}
  \usepackage[utf8]{inputenc}
  \usepackage{textcomp} % provide euro and other symbols
\else % if luatex or xetex
  \usepackage{unicode-math}
  \defaultfontfeatures{Scale=MatchLowercase}
  \defaultfontfeatures[\rmfamily]{Ligatures=TeX,Scale=1}
\fi
% Use upquote if available, for straight quotes in verbatim environments
\IfFileExists{upquote.sty}{\usepackage{upquote}}{}
\IfFileExists{microtype.sty}{% use microtype if available
  \usepackage[]{microtype}
  \UseMicrotypeSet[protrusion]{basicmath} % disable protrusion for tt fonts
}{}
\makeatletter
\@ifundefined{KOMAClassName}{% if non-KOMA class
  \IfFileExists{parskip.sty}{%
    \usepackage{parskip}
  }{% else
    \setlength{\parindent}{0pt}
    \setlength{\parskip}{6pt plus 2pt minus 1pt}}
}{% if KOMA class
  \KOMAoptions{parskip=half}}
\makeatother
\usepackage{xcolor}
\IfFileExists{xurl.sty}{\usepackage{xurl}}{} % add URL line breaks if available
\IfFileExists{bookmark.sty}{\usepackage{bookmark}}{\usepackage{hyperref}}
\hypersetup{
  pdftitle={PANDOC(1) Pandoc User Manuals},
  pdfauthor={John MacFarlane},
  hidelinks,
  pdfcreator={LaTeX via pandoc}}
\urlstyle{same} % disable monospaced font for URLs
\usepackage[margin=3cm]{geometry}
\setlength{\emergencystretch}{3em} % prevent overfull lines
\providecommand{\tightlist}{%
  \setlength{\itemsep}{0pt}\setlength{\parskip}{0pt}}
\setcounter{secnumdepth}{-\maxdimen} % remove section numbering

\title{PANDOC(1) Pandoc User Manuals}
\author{John MacFarlane}
\date{January 8, 2008}

\begin{document}
\maketitle

\setstretch{1.25}
\hypertarget{name}{%
\section{NAME}\label{name}}

pandoc - general markup converter

\hypertarget{synopsis}{%
\section{SYNOPSIS}\label{synopsis}}

pandoc {[}\emph{options}{]} {[}\emph{input-file}{]}\ldots{}

\hypertarget{description}{%
\section{DESCRIPTION}\label{description}}

Pandoc converts files from one markup format to another. It can read markdown
and (subsets of) reStructuredText, HTML, and LaTeX, and it can write plain
text, markdown, reStructuredText, HTML, LaTeX, ConTeXt, Texinfo, groff man,
MediaWiki markup, RTF, OpenDocument XML, ODT, DocBook XML, EPUB, and Slidy or
S5 HTML slide shows.

If no \emph{input-file} is specified, input is read from \emph{stdin}.
Otherwise, the \emph{input-files} are concatenated (with a blank line between
each) and used as input. Output goes to \emph{stdout} by default (though
output to \emph{stdout} is disabled for the \texttt{odt} and \texttt{epub}
output formats). For output to a file, use the \texttt{-o} option:

\begin{verbatim}
pandoc -o output.html input.txt
\end{verbatim}

Instead of a file, an absolute URI may be given. In this case pandoc will
fetch the content using HTTP:

\begin{verbatim}
pandoc -f html -t markdown http://www.fsf.org
\end{verbatim}

The input and output formats may be specified using command-line options (see
\textbf{OPTIONS}, below, for details). If these formats are not specified
explicitly, Pandoc will attempt to determine them from the extensions of the
input and output filenames. If input comes from \emph{stdin} or from a file
with an unknown extension, the input is assumed to be markdown. If no output
filename is specified using the \texttt{-o} option, or if a filename is
specified but its extension is unknown, the output will default to HTML. Thus,
for example,

\begin{verbatim}
pandoc -o chap1.tex chap1.txt
\end{verbatim}

converts \emph{chap1.txt} from markdown to LaTeX. And

\begin{verbatim}
pandoc README
\end{verbatim}

converts \emph{README} from markdown to HTML.

Pandoc's version of markdown is an extended variant of standard markdown: the
differences are described in the \emph{README} file in the user documentation.
If standard markdown syntax is desired, the \texttt{-\/-strict} option may be
used.

Pandoc uses the UTF-8 character encoding for both input and output. If your
local character encoding is not UTF-8, you should pipe input and output
through \texttt{iconv}:

\begin{verbatim}
iconv -t utf-8 input.txt | pandoc | iconv -f utf-8
\end{verbatim}

\hypertarget{options}{%
\section{OPTIONS}\label{options}}

\begin{description}
\item[-f \emph{FORMAT}, -r \emph{FORMAT}, -\/-from=\emph{FORMAT},
-\/-read=\emph{FORMAT}]
Specify input format. \emph{FORMAT} can be \texttt{native} (native Haskell),
\texttt{markdown} (markdown or plain text), \texttt{rst} (reStructuredText),
\texttt{html} (HTML), or \texttt{latex} (LaTeX). If \texttt{+lhs} is appended
to \texttt{markdown}, \texttt{rst}, or \texttt{latex}, the input will be
treated as literate Haskell source.
\item[-t \emph{FORMAT}, -w \emph{FORMAT}, -\/-to=\emph{FORMAT},
-\/-write=\emph{FORMAT}]
Specify output format. \emph{FORMAT} can be \texttt{native} (native Haskell),
\texttt{plain} (plain text), \texttt{markdown} (markdown), \texttt{rst}
(reStructuredText), \texttt{html} (HTML), \texttt{latex} (LaTeX),
\texttt{context} (ConTeXt), \texttt{man} (groff man), \texttt{mediawiki}
(MediaWiki markup), \texttt{texinfo} (GNU Texinfo), \texttt{docbook} (DocBook
XML), \texttt{opendocument} (OpenDocument XML), \texttt{odt} (OpenOffice text
document), \texttt{epub} (EPUB book), \texttt{slidy} (Slidy HTML and
javascript slide show), \texttt{s5} (S5 HTML and javascript slide show), or
\texttt{rtf} (rich text format). Note that \texttt{odt} and \texttt{epub}
output will not be directed to \emph{stdout}; an output filename must be
specified using the \texttt{-o/-\/-output} option. If \texttt{+lhs} is
appended to \texttt{markdown}, \texttt{rst}, \texttt{latex}, or \texttt{html},
the output will be rendered as literate Haskell source.
\item[-s, -\/-standalone]
Produce output with an appropriate header and footer (e.g.~a standalone HTML,
LaTeX, or RTF file, not a fragment).
\item[-o \emph{FILE}, -\/-output=\emph{FILE}]
Write output to \emph{FILE} instead of \emph{stdout}. If \emph{FILE} is
`\texttt{-}', output will go to \emph{stdout}.
\item[-p, -\/-preserve-tabs]
Preserve tabs instead of converting them to spaces.
\item[-\/-tab-stop=\emph{TABSTOP}]
Specify tab stop (default is 4).
\item[-\/-strict]
Use strict markdown syntax, with no extensions or variants.
\item[-\/-reference-links]
Use reference-style links, rather than inline links, in writing markdown or
reStructuredText.
\item[-R, -\/-parse-raw]
Parse untranslatable HTML codes and LaTeX environments as raw HTML or LaTeX,
instead of ignoring them.
\item[-S, -\/-smart]
Use smart quotes, dashes, and ellipses. (This option is significant only when
the input format is \texttt{markdown}. It is selected automatically when the
output format is \texttt{latex} or \texttt{context}.)
\item[-m\emph{URL}, -\/-latexmathml=\emph{URL}]
Use LaTeXMathML to display embedded TeX math in HTML output. To insert a link
to a local copy of the \texttt{LaTeXMathML.js} script, provide a \emph{URL}.
If no \emph{URL} is provided, the contents of the script will be inserted
directly into the HTML header.
\item[-\/-mathml]
Convert TeX math to MathML. In standalone mode, a small javascript will be
inserted that allows the MathML to be viewed on some browsers.
\item[-\/-jsmath=\emph{URL}]
Use jsMath to display embedded TeX math in HTML output. The \emph{URL} should
point to the jsMath load script; if provided, it will be linked to in the
header of standalone HTML documents.
\item[-\/-gladtex]
Enclose TeX math in \texttt{\textless{}eq\textgreater{}} tags in HTML output.
These can then be processed by gladTeX to produce links to images of the
typeset formulas.
\item[-\/-mimetex=\emph{URL}]
Render TeX math using the mimeTeX CGI script. If \emph{URL} is not specified,
it is assumed that the script is at \texttt{/cgi-bin/mimetex.cgi}.
\item[-\/-webtex=\emph{URL}]
Render TeX math using an external script. The formula will be concatenated
with the URL provided. If \emph{URL} is not specified, the Google Chart API
will be used.
\item[-i, -\/-incremental]
Make list items in Slidy or S5 display incrementally (one by one).
\item[-\/-offline]
Include all the CSS and javascript needed for a Slidy or S5 slide show in the
output, so that the slide show will work even when no internet connection is
available.
\item[-\/-xetex]
Create LaTeX outut suitable for processing by XeTeX.
\item[-N, -\/-number-sections]
Number section headings in LaTeX, ConTeXt, or HTML output. (Default is not to
number them.)
\item[-\/-section-divs]
Wrap sections in \texttt{\textless{}div\textgreater{}} tags, and attach
identifiers to the enclosing \texttt{\textless{}div\textgreater{}} rather than
the header itself.
\item[-\/-no-wrap]
Disable text wrapping in output. (Default is to wrap text.)
\item[-\/-sanitize-html]
Sanitizes HTML (in markdown or HTML input) using a whitelist. Unsafe tags are
replaced by HTML comments; unsafe attributes are omitted. URIs in links and
images are also checked against a whitelist of URI schemes.
\item[-\/-email-obfuscation=\emph{none\textbar javascript\textbar references}]
Specify a method for obfuscating \texttt{mailto:} links in HTML documents.
\emph{none} leaves \texttt{mailto:} links as they are. \emph{javascript}
obfuscates them using javascript. \emph{references} obfuscates them by
printing their letters as decimal or hexadecimal character references. If
\texttt{-\/-strict} is specified, \emph{references} is used regardless of the
presence of this option.
\item[-\/-id-prefix\emph{=string}]
Specify a prefix to be added to all automatically generated identifiers in
HTML output. This is useful for preventing duplicate identifiers when
generating fragments to be included in other pages.
\item[-\/-indented-code-classes\emph{=classes}]
Specify classes to use for indented code blocks--for example,
\texttt{perl,numberLines} or \texttt{haskell}. Multiple classes may be
separated by spaces or commas.
\item[-\/-toc, -\/-table-of-contents]
Include an automatically generated table of contents (HTML, markdown, RTF) or
an instruction to create one (LaTeX, reStructuredText). This option has no
effect on man, DocBook, Slidy, or S5 output.
\item[-\/-base-header-level=\emph{LEVEL}]
Specify the base level for headers (defaults to 1).
\item[-\/-template=\emph{FILE}]
Use \emph{FILE} as a custom template for the generated document. Implies
\texttt{-s}. See TEMPLATES below for a description of template syntax. If this
option is not used, a default template appropriate for the output format will
be used. See also \texttt{-D/-\/-print-default-template}.
\item[-V KEY=VAL, -\/-variable=\emph{KEY:VAL}]
Set the template variable KEY to the value VAL when rendering the document in
standalone mode. This is only useful when the \texttt{-\/-template} option is
used to specify a custom template, since pandoc automatically sets the
variables used in the default templates.
\item[-c \emph{CSS}, -\/-css=\emph{CSS}]
Link to a CSS style sheet. \emph{CSS} is the pathname of the style sheet.
\item[-H \emph{FILE}, -\/-include-in-header=\emph{FILE}]
Include contents of \emph{FILE} at the end of the header. Implies \texttt{-s}.
\item[-B \emph{FILE}, -\/-include-before-body=\emph{FILE}]
Include contents of \emph{FILE} at the beginning of the document body. Implies
\texttt{-s}.
\item[-A \emph{FILE}, -\/-include-after-body=\emph{FILE}]
Include contents of \emph{FILE} at the end of the document body. Implies
\texttt{-s}.
\item[-C \emph{FILE}, -\/-custom-header=\emph{FILE}]
Use contents of \emph{FILE} as the document header. \emph{Note: This option is
deprecated. Users should transition to using \texttt{-\/-template} instead.}
\item[-\/-reference-odt=\emph{filename}]
Use the specified file as a style reference in producing an ODT. For best
results, the reference ODT should be a modified version of an ODT produced
using pandoc. The contents of the reference ODT are ignored, but its
stylesheets are used in the new ODT. If no reference ODT is specified on the
command line, pandoc will look for a file \texttt{reference.odt} in the user
data directory (see \texttt{-\/-data-dir}). If this is not found either,
sensible defaults will be used.
\item[-\/-epub-stylesheet=\emph{filename}]
Use the specified CSS file to style the EPUB. If no stylesheet is specified,
pandoc will look for a file \texttt{epub.css} in the user data directory (see
\texttt{-\/-data-dir}, below). If it is not found there, sensible defaults
will be used.
\item[-\/-epub-metadata=\emph{filename}]
Look in the specified XML file for metadata for the EPUB. The file should
contain a series of Dublin Core elements
(http://dublincore.org/documents/dces/), for example:

\begin{verbatim}
 <dc:rights>Creative Commons</dc:rights>
 <dc:language>es-AR</dc:language>
\end{verbatim}

By default, pandoc will include the following metadata elements:
\texttt{\textless{}dc:title\textgreater{}} (from the document title),
\texttt{\textless{}dc:creator\textgreater{}} (from the document authors),
\texttt{\textless{}dc:language\textgreater{}} (from the locale), and
\texttt{\textless{}dc:identifier\ id="BookId"\textgreater{}} (a randomly
generated UUID). Any of these may be overridden by elements in the metadata
file.
\item[-D \emph{FORMAT}, -\/-print-default-template=\emph{FORMAT}]
Print the default template for an output \emph{FORMAT}. (See \texttt{-t} for a
list of possible \emph{FORMAT}s.)
\item[-T \emph{STRING}, -\/-title-prefix=\emph{STRING}]
Specify \emph{STRING} as a prefix to the HTML window title.
\item[-\/-data-dir\emph{=DIRECTORY}]
Specify the user data directory to search for pandoc data files. If this
option is not specified, the default user data directory will be used:

\begin{verbatim}
$HOME/.pandoc
\end{verbatim}

in unix and

\begin{verbatim}
C:\Documents And Settings\USERNAME\Application Data\pandoc
\end{verbatim}

in Windows. A \texttt{reference.odt}, \texttt{epub.css}, \texttt{templates}
directory, or \texttt{s5} directory placed in this directory will override
pandoc's normal defaults.
\item[-\/-dump-args]
Print information about command-line arguments to \emph{stdout}, then exit.
The first line of output contains the name of the output file specified with
the \texttt{-o} option, or `\texttt{-}' (for \emph{stdout}) if no output file
was specified. The remaining lines contain the command-line arguments, one per
line, in the order they appear. These do not include regular Pandoc options
and their arguments, but do include any options appearing after a
`\texttt{-\/-}' separator at the end of the line. This option is intended
primarily for use in wrapper scripts.
\item[-\/-ignore-args]
Ignore command-line arguments (for use in wrapper scripts). Regular Pandoc
options are not ignored. Thus, for example,

\begin{verbatim}
pandoc --ignore-args -o foo.html -s foo.txt -- -e latin1
\end{verbatim}

is equivalent to

\begin{verbatim}
pandoc -o foo.html -s
\end{verbatim}
\item[-v, -\/-version]
Print version.
\item[-h, -\/-help]
Show usage message.
\end{description}

\hypertarget{templates}{%
\section{TEMPLATES}\label{templates}}

When the \texttt{-s/-\/-standalone} option is used, pandoc uses a template to
add header and footer material that is needed for a self-standing document. To
see the default template that is used, just type

\begin{verbatim}
pandoc --print-default-template=FORMAT
\end{verbatim}

where \texttt{FORMAT} is the name of the output format. A custom template can
be specified using the \texttt{-\/-template} option. You can also override the
system default templates for a given output format \texttt{FORMAT} by putting
a file \texttt{templates/FORMAT.template} in the user data directory (see
\texttt{-\/-data-dir}, below).

Templates may contain \emph{variables}. Variable names are sequences of
alphanumerics, \texttt{-}, and \texttt{\_}, starting with a letter. A variable
name surrounded by \texttt{\$} signs will be replaced by its value. For
example, the string \texttt{\$title\$} in

\begin{verbatim}
<title>$title$</title>
\end{verbatim}

will be replaced by the document title.

To write a literal \texttt{\$} in a template, use \texttt{\$\$}.

Some variables are set automatically by pandoc. These vary somewhat depending
on the output format, but include:

\begin{description}
\tightlist
\item[\texttt{legacy-header}]
contents specified by \texttt{-C/-\/-custom-header} \texttt{header-includes}

contents specified by \texttt{-H/-\/-include-in-header} (may have multiple
values) \texttt{toc}

non-null value if \texttt{-\/-toc/-\/-table-of-contents} was specified
\texttt{include-before}

contents specified by \texttt{-B/-\/-include-before-body} (may have multiple
values) \texttt{include-after}

contents specified by \texttt{-A/-\/-include-after-body} (may have multiple
values) \texttt{body}

body of document \texttt{title}

title of document, as specified in title block \texttt{author}

author of document, as specified in title block (may have multiple values)
\texttt{date}

date of document, as specified in title block
\end{description}

Variables may be set at the command line using the \texttt{-V/-\/-variable}
option. This allows users to include custom variables in their templates.

Templates may contain conditionals. The syntax is as follows:

\begin{verbatim}
$if(variable)$
X 
$else$
Y
$endif$
\end{verbatim}

This will include \texttt{X} in the template if \texttt{variable} has a
non-null value; otherwise it will include \texttt{Y}. \texttt{X} and
\texttt{Y} are placeholders for any valid template text, and may include
interpolated variables or other conditionals. The \texttt{\$else\$} section
may be omitted.

When variables can have multiple values (for example, \texttt{author} in a
multi-author document), you can use the \texttt{\$for\$} keyword:

\begin{verbatim}
$for(author)$
<meta name="author" content="$author$" />
$endfor$
\end{verbatim}

You can optionally specify a separator to be used between consecutive items:

\begin{verbatim}
$for(author)$$author$$sep$, $endfor$
\end{verbatim}

\hypertarget{see-also}{%
\section{SEE ALSO}\label{see-also}}

\texttt{markdown2pdf} (1). The \emph{README} file distributed with Pandoc
contains full documentation.

The Pandoc source code and all documentation may be downloaded from
\url{http://johnmacfarlane.net/pandoc/}.

\end{document}
